\hypertarget{introduction}{%
\section{Introduction}\label{introduction}}

The aim of this tutorial is to teach people somewhat familiar with the
Soar cognitive architecture how to write a Soar agent to control a Cozmo
robot. It assumes that Soar is already installed on your computer and
that you have the Soar-Cozmo Interface code ready to be run. Files
associated with each part of the tutorial are included in this
directory, and should be able to be sourced presently. When reading a
lesson, I strongly suggest opening the associated \texttt{.soar} file
and keeping it handy so that you can reference it while reading the
tutorial. Ideally, by the end of the tutorial, you should understand
what the inputs made available to Soar by Cozmo are and what they do, as
well aome of the commands Soar can output to Cozmo and what these do.
Although the tutorial will not cover every available Cozmo action
currently supported, I hope it will cover enough to give you the context
and background needed to understand the rest through inspection of the
documentation and the interface code. Before getting into examples, we
will first go over the higher-level mechanics of how the interface
works.

\hypertarget{how-it-works}{%
\section{How It Works}\label{how-it-works}}

First, it is important to note that the interface code is purely
\emph{reactive} to the Soar kernel, which is considered primary. That
means the Soar kernel and its cycle is what drives the interface, rather
than any internal loop in the interface itself. All of the functionality
of the interface is within various \emph{callback} functions, which are
triggered by the Soar kernel, execute their code, and then return
control to the kernel. Two callback functions in particular are used to
provide the two primary functions of the interface: providing
information about the Cozmo and its perceptions on the input link and
watching the output link for actions the agent wants Cozmo to perform.

\hypertarget{input}{%
\subsection{Input}\label{input}}

The input is achieved by a callback triggered when the Soar cycle is
about to enter the input phase. When the cycle reaches that point, the
interface polls the Cozmo robot for all the information which needs to
be presented to Soar such as its pose, the status of its lift, and any
objects or faces it can see. We will cover the details of how this
information is presented later. Once it has the appropriate information
from Cozmo, the interface updates the agent's working memory elements
through the Soar Markup Language, and passes control back to the kernel,
which proceeds into the input phase. By updating the working memory
elements before the input phase, we ensure that the agent has the latest
information available.

\hypertarget{output}{%
\subsection{Output}\label{output}}

Control of the Cozmo robot is achieved by a callback which listens for
changes to the output link, then scans it for new working memory
elements with names of actions Cozmo can take. For example, if the
output link is initially empty and Soar adds a new \texttt{move-lift}
identifier \texttt{M1} which has its own attribute \texttt{height} set
to 0, the new output link would look like

\begin{verbatim}
(I3 ^move-lift M1)
  (M2 ^height 0.5)
\end{verbatim}

and the output callback would be triggered. It would find the new output
link attribute \texttt{\^{}move-lift} and its associated identifier, and
then see if the string ``move-lift'' is an action it recognizes. Since
it is, it will look at the identifier \texttt{M1} and try and find the
``height'' attribute. The value of the ``height'' attribute is used in a
function call to the Cozmo SDK which will move the lift to the specified
level, in this case 50\% of its maximum height. Once the Cozmo finishes
the action, a ``status'' attribute is added to the action's identifier
and set to ``complete'', so the final output link looks like

\begin{verbatim}
(I3 ^move-lift M1)
  (M1 ^height 0.5 ^status complete)
\end{verbatim}

Control is then handed back to the Soar kernel, which continues to run
its cycle. If more than one new valid action is added to the output link
Cozmo will execute all of them before handing back control. The order of
execution should be treated as random.

\hypertarget{lessons}{%
\section{Lessons}\label{lessons}}

To ground that rather abstract description of the interface, and to
provide examples of the specific inputs and outputs the interface
provides and handles, we will go through a series of lessons which
incrementally build on each other, ultimately producing a Soar agent
which will look for a face, then a light cube, then try and bring the
light cube to the face.

\hypertarget{lesson-1-reset-cozmos-head-and-lift}{%
\subsection{Lesson 1: Reset Cozmo's Head and
Lift}\label{lesson-1-reset-cozmos-head-and-lift}}

First, we'll explore how to move Cozmo's head and lift to specific
positions using the \texttt{move-head} and \texttt{move-lift} commands.
The \texttt{move-head} action sets the angle on Cozmo's head relative to
a horizontal plane through its axis of rotation. Thus, if the angle
specified is -0.25, Cozmo's head will move so that it forms a -0.25
radian angle with the plane, which ends up having Cozmo look towards the
ground. An angle of 0.25 will similarly have Cozmo look upwards. The
\texttt{move-lift} action moves the lift to a position specified in the
command by a real number between 0 and 1. The number indicates the
percentage of the lifts maximum height it should move to, so a value of
0 is the lowest possible height for the lift, a value of 1 is the
highest, and 0.5 is the lift's midpoint.

We will be using the \texttt{move-head} and \texttt{move-lift} actions
to reset Cozmo's head and lift positions to default ones. Specifically,
we want Cozmo to move its head to be parallel with the ground and to
lower its lift as far as it can. Together, these actions will help Cozmo
see better, since Cozmo often starts with its tilted down, and the lift
can occasionally block Cozmo's camera. In order to make sure Cozmo
always resets when the Soar agent starts, we will make a slight addition
to the usual initialization production for a Soar agent. The
initialization proposal rule is a cookie-cutter proposal which just
proposes the \texttt{initialize-cozmo} operator. The application rule
checks for the presence of this operator, and then has two parts. The
first is fairly standard: it adds a \texttt{\^{}name} attribute to the
top-state and sets its value to \texttt{cozmo}. The second is what
resets the positions of Cozmo's lift and head:

\begin{verbatim}
(<out>  ^move-head.angle 0.0
        ^move-lift.height 0.0)
\end{verbatim}

This part of the right hand side (RHS) adds two working memory elements
to the output link, \texttt{\^{}move-head} and \texttt{move-lift}, and
gives them each an attribute, \texttt{\^{}angle} for
\texttt{\^{}move-head} and \texttt{\^{}height} for \texttt{move-lift},
which are set to 0.0. Recall that the interface listens for new
additions to the output link. This means that when this rule fires, the
interface will pause the kernel to look for valid actions, which both
new WMEs are. The interface will be looking for an ``angle'' attribute
on the ``move-head'' WME and a ``height'' attribute on the ``move-lift''
WME. Since both are present and supply valid values, the interface will
execute the specified actions in the Cozmo robot.

\hypertarget{lesson-2-saving-the-origin-pose}{%
\subsection{Lesson 2: Saving the Origin
Pose}\label{lesson-2-saving-the-origin-pose}}

The first input we will be looking at will be Cozmo's pose information,
which is always placed by the interface on the agent's input link and
will look similar to:

\begin{verbatim}
(I2 ^pose P1)
    (P1 ^rot 2.733525 ^x 31.371500 ^y 1.045640 ^z 0.000000)
\end{verbatim}

The pose identifier will have four attributes with floating point
values, \texttt{rot}, \texttt{x}, \texttt{y}, and \texttt{z}, indicating
Cozmo's rotation on the z (vertical) axis in radians and its position on
the x-y plane from the origin in millimeters. Although these values are
just estimations based on Cozmo's internal sensors, they are
never-the-less useful in keeping track of where Cozmo is. Right now, we
are going to make sure that Cozmo also keeps track of where it started
by saving Cozmo's initial pose when it starts up. This will involve
modifying both the proposal and the application rule we touched on in
the last lesson.

First, we need to modify the proposal rule so that the
\texttt{initialize-cozmo} operator has the initial pose information. In
the left hand side (LHS), add a new attribute to search for on the top
state, \texttt{io.input-link.pose} and assign it ot variable
\texttt{\textless{}p\textgreater{}}, like so:

\begin{verbatim}
(state <s>  ^superstate nil
           -^name
            ^io.input-link.pose <p>)
\end{verbatim}

Then expand \texttt{\textless{}p\textgreater{}} out to get the actual
values of the pose in a separate conditional:

\begin{verbatim}
(<p> ^rot <rot>
     ^x <x-val>
     ^y <y-val>
     ^z <z-val>)
\end{verbatim}

This will look at the pose information on the agent's input link and
store the values in the variables \texttt{\textless{}rot\textgreater{}}
(for ``rotation''), \texttt{\textless{}x-val\textgreater{}},
\texttt{\textless{}y-val\textgreater{}}, and
\texttt{\textless{}z-val\textgreater{}}, so that we can attach them to
the operator.

On the RHS, add a new attribute \texttt{origin} to the operator
\texttt{\textless{}op\textgreater{}} so it looks like:

\begin{verbatim}
(<op>   ^name initialize-cozmo
        ^origin <ogn>)
\end{verbatim}

and then expand \texttt{\textless{}ogn\textgreater{}}, adding
\texttt{rot}, \texttt{x}, \texttt{y}, and \texttt{z} attributes:

\begin{verbatim}
(<ogn>  ^rot <rot>
        ^x <x-val>
        ^y <y-val>
        ^z <z-val>)
\end{verbatim}

Now the operator has the pose information we want to store, which means
the application rule an pull directly from the operator. Now we need to
modify the application rule so that it actually stores the origin pose
on the top-state. First, modify its LHS a bit by looking for an
\texttt{origin} attribute to the operator and storing it in the variable
\texttt{\textless{}ogn\textgreater{}}. Then, just like above, expand out
\texttt{\textless{}ogn\textgreater{}} so that the agent stores the
rotation and location information in
\texttt{\textless{}rot\textgreater{}},
\texttt{\textless{}x-val\textgreater{}},
\texttt{\textless{}y-val\textgreater{}}, and
\texttt{\textless{}z-val\textgreater{}}. On the RHS, create an
\texttt{origin} attribute on the state
\texttt{\textless{}s\textgreater{}}, like so:

\begin{verbatim}
(<s>    ^name cozmo
        ^origin <origin>)
\end{verbatim}

Then expand it out like we have before by adding a new effect:

\begin{verbatim}
(<origin>   ^rot <rot>
            ^x <x-val>
            ^y <y-val>
            ^z <z-val>)
\end{verbatim}

Note that \texttt{origin} is stored directly on the top state, rather
than in the input link, because it is not an input but a memory.
Additionally, because we check for an operator in the LHS, the
\texttt{origin} WME is \emph{o-supported}, meaning it will persist even
after the rule which added it (\texttt{apply*initialize-cozmo}) is
retracted. Thus, we have a permanent record of where Cozmo started.
